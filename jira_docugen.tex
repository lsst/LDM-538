\section{Test Cases Summary}\label{test-cases-summary}

Follows the list of test cases documented in this specification.

\begin{longtable}[]{p{3cm}p{13cm}}
\toprule
Test Id & Test Name\tabularnewline
\midrule
\endhead
    \hyperref[lvv-t283]{LVV-T283} &
    \href{https://jira.lsstcorp.org/secure/Tests.jspa\#/testCase/LVV-T283}{RAS-00-00: Writing well-formed raw image} \tabularnewline
    \hyperref[lvv-t284]{LVV-T284} &
    \href{https://jira.lsstcorp.org/secure/Tests.jspa\#/testCase/LVV-T284}{RAS-00-05:Writing data from CCOB to the DBB for further data processing} \tabularnewline
    \hyperref[lvv-t285]{LVV-T285} &
    \href{https://jira.lsstcorp.org/secure/Tests.jspa\#/testCase/LVV-T285}{RAS-00-10: Raw images in Observatory Operations Data Service} \tabularnewline
    \hyperref[lvv-t286]{LVV-T286} &
    \href{https://jira.lsstcorp.org/secure/Tests.jspa\#/testCase/LVV-T286}{RAS-00-20: Raw image are part of the permanent record of survey via DBB} \tabularnewline
    \hyperref[lvv-t287]{LVV-T287} &
    \href{https://jira.lsstcorp.org/secure/Tests.jspa\#/testCase/LVV-T287}{RAS-00-30: Raw Image Archiving Availability, Throughput, Reliability, and Heterogeneity} \tabularnewline
\bottomrule
\end{longtable}

\newpage

\section{Test Cases}

\subsection{\href{https://jira.lsstcorp.org/secure/Tests.jspa\#/testCase/LVV-T283}{LVV-T283}
    - RAS-00-00: Writing well-formed raw image}\label{lvv-t283}

\begin{longtable}[]{llllll}
\toprule
Version & Status & Priority & Verification Type & Critical Event & Owner
\\\midrule
1 & Draft & Normal &
Test & False & Michelle Butler
\\\bottomrule
\end{longtable}

\subsubsection{Requirements}
\begin{itemize}
\item \href{https://jira.lsstcorp.org/browse/LVV-8}{LVV-8} - DMS-REQ-0018-V-01: Raw Science Image Data Acquisition
\item \href{https://jira.lsstcorp.org/browse/LVV-9}{LVV-9} - DMS-REQ-0020-V-01: Wavefront Sensor Data Acquisition
\item \href{https://jira.lsstcorp.org/browse/LVV-96}{LVV-96} - DMS-REQ-0265-V-01: Guider Calibration Data Acquisition
\item \href{https://jira.lsstcorp.org/browse/LVV-28}{LVV-28} - DMS-REQ-0068-V-01: Raw Science Image Metadata
\item \href{https://jira.lsstcorp.org/browse/LVV-11}{LVV-11} - DMS-REQ-0024-V-01: Raw Image Assembly
\item \href{https://jira.lsstcorp.org/browse/LVV-146}{LVV-146} - DMS-REQ-0315-V-01: DMS Communication with OCS
\item \href{https://jira.lsstcorp.org/browse/LVV-115}{LVV-115} - DMS-REQ-0284-V-01: Level-1 Production Completeness
\end{itemize}

\subsubsection{Test Items}
This test will check:\\

\begin{itemize}
\tightlist
\item
  The successful integration of the Pathfinder components with the DM
  Header Service and the Level 1 Archiver;
\item
  That the raw images are well-formed and meet specifications in
  change-controlled documents LSE-61;
\end{itemize}

~This Test Case shall be repeated for each of the different cameras
(ATScam, LSSTCam) and sensors (Science, Wavefront, and Guider)
combination.



\subsubsection{Intercase Dependencies}
None.


\subsubsection{Environment Needs}

\paragraph{Software}
\begin{itemize}
\tightlist
\item
  Level 1 software and services needed to create raw image
\item
  LSST Monitoring Service and plugins specific to monitoring Level 1
  Test Stand and services
\end{itemize}


\paragraph{Hardware}
\begin{itemize}
\tightlist
\item
  Level 1 test stand
\item
  Test machine for LSST Monitoring Service
\end{itemize}


\subsubsection{Input Specification}
None.


\subsubsection{Output Specification}
Raw image(s) that follow specifications defined in change-controlled
document LSE-61.


\subsubsection{Test Procedure}
    \begin{longtable}[]{p{1.3cm}p{2cm}p{13cm}}
    %\toprule
    Step & \multicolumn{2}{@{}l}{Description, Input Data and Expected Result} \\ \toprule
    \endhead

            \multirow{3}{*}{ 1 } & Description &
            \begin{minipage}[t]{13cm}{\footnotesize
            Configure system to pull appropriate data from the DAQ emulator

            \vspace{\dp0}
            } \end{minipage} \\ \cline{2-3}
            & Test Data &
            \begin{minipage}[t]{13cm}{\footnotesize
                No data.
                \vspace{\dp0}
            } \end{minipage} \\ \cline{2-3}
            & Expected Result &
                \begin{minipage}[t]{13cm}{\footnotesize
                A functional DAQ for images to be received from.~~

                \vspace{\dp0}
                } \end{minipage}
        \\ \midrule

            \multirow{3}{*}{ 2 } & Description &
            \begin{minipage}[t]{13cm}{\footnotesize
            Acquire raw data from DAQ readout and DMHS

            \vspace{\dp0}
            } \end{minipage} \\ \cline{2-3}
            & Test Data &
            \begin{minipage}[t]{13cm}{\footnotesize
                No data.
                \vspace{\dp0}
            } \end{minipage} \\ \cline{2-3}
            & Expected Result &
                \begin{minipage}[t]{13cm}{\footnotesize
                a raw image and a header from the DMHS~

                \vspace{\dp0}
                } \end{minipage}
        \\ \midrule

            \multirow{3}{*}{ 3 } & Description &
            \begin{minipage}[t]{13cm}{\footnotesize
            Fetch data and reassemble correctly, regardless of CCD/Sensor
manufacturer type (two different types will be used)

            \vspace{\dp0}
            } \end{minipage} \\ \cline{2-3}
            & Test Data &
            \begin{minipage}[t]{13cm}{\footnotesize
                No data.
                \vspace{\dp0}
            } \end{minipage} \\ \cline{2-3}
            & Expected Result &
                \begin{minipage}[t]{13cm}{\footnotesize
                Build the data into a fits file

                \vspace{\dp0}
                } \end{minipage}
        \\ \midrule

            \multirow{3}{*}{ 4 } & Description &
            \begin{minipage}[t]{13cm}{\footnotesize
            CheckCompletenessandcorrectnessoftherawimagesincludingformat,metadata,and
image data;

\begin{itemize}
\tightlist
\item
  Check proper fetch and reassembly of image data from camera DAQ
  (correct format and data);
\item
  Check proper merge of header service data with image data;
\item
  Check correct insertion of exposure specific data needed in the data
  file that is not supplied by header service;
\item
  Check minimum required metadata (from requirements document LSE-61)
  exists in raw image header;
\end{itemize}

            \vspace{\dp0}
            } \end{minipage} \\ \cline{2-3}
            & Test Data &
            \begin{minipage}[t]{13cm}{\footnotesize
                No data.
                \vspace{\dp0}
            } \end{minipage} \\ \cline{2-3}
            & Expected Result &
                \begin{minipage}[t]{13cm}{\footnotesize
                a well formed FITS file with a proper header that has been verified to
be correct.~

                \vspace{\dp0}
                } \end{minipage}
        \\ \midrule

            \multirow{3}{*}{ 5 } & Description &
            \begin{minipage}[t]{13cm}{\footnotesize
            Check that the checksum of the file matches the previously calculated
value that will be passed on to downstream services

            \vspace{\dp0}
            } \end{minipage} \\ \cline{2-3}
            & Test Data &
            \begin{minipage}[t]{13cm}{\footnotesize
                No data.
                \vspace{\dp0}
            } \end{minipage} \\ \cline{2-3}
            & Expected Result &
                \begin{minipage}[t]{13cm}{\footnotesize
                a MD5sum number generated from the step 4 file.~~

                \vspace{\dp0}
                } \end{minipage}
        \\ \midrule

            \multirow{3}{*}{ 6 } & Description &
            \begin{minipage}[t]{13cm}{\footnotesize
            Check confirmation that the data files arrive at their destination
intact

            \vspace{\dp0}
            } \end{minipage} \\ \cline{2-3}
            & Test Data &
            \begin{minipage}[t]{13cm}{\footnotesize
                No data.
                \vspace{\dp0}
            } \end{minipage} \\ \cline{2-3}
            & Expected Result &
                \begin{minipage}[t]{13cm}{\footnotesize
                a transfer of the file to the correct location for further retrieval
from other services.~~

                \vspace{\dp0}
                } \end{minipage}
        \\ \midrule

            \multirow{3}{*}{ 7 } & Description &
            \begin{minipage}[t]{13cm}{\footnotesize
            Check that LSST Monitoring Service showed the appropriate information
successfully

            \vspace{\dp0}
            } \end{minipage} \\ \cline{2-3}
            & Test Data &
            \begin{minipage}[t]{13cm}{\footnotesize
                No data.
                \vspace{\dp0}
            } \end{minipage} \\ \cline{2-3}
            & Expected Result &
                \begin{minipage}[t]{13cm}{\footnotesize
                all systems remained green through out the test, and showed all systems
up and available. ~\\
~\\

                \vspace{\dp0}
                } \end{minipage}
        \\ \midrule
    \end{longtable}

\subsection{\href{https://jira.lsstcorp.org/secure/Tests.jspa\#/testCase/LVV-T284}{LVV-T284}
    - RAS-00-05:Writing data from CCOB to the DBB for further data processing}\label{lvv-t284}

\begin{longtable}[]{llllll}
\toprule
Version & Status & Priority & Verification Type & Critical Event & Owner
\\\midrule
1 & Draft & Normal &
Test & False & Michelle Butler
\\\bottomrule
\end{longtable}

\subsubsection{Requirements}
\begin{itemize}
\item \href{https://jira.lsstcorp.org/browse/LVV-9}{LVV-9} - DMS-REQ-0020-V-01: Wavefront Sensor Data Acquisition
\item \href{https://jira.lsstcorp.org/browse/LVV-8}{LVV-8} - DMS-REQ-0018-V-01: Raw Science Image Data Acquisition
\item \href{https://jira.lsstcorp.org/browse/LVV-96}{LVV-96} - DMS-REQ-0265-V-01: Guider Calibration Data Acquisition
\item \href{https://jira.lsstcorp.org/browse/LVV-28}{LVV-28} - DMS-REQ-0068-V-01: Raw Science Image Metadata
\item \href{https://jira.lsstcorp.org/browse/LVV-11}{LVV-11} - DMS-REQ-0024-V-01: Raw Image Assembly
\item \href{https://jira.lsstcorp.org/browse/LVV-146}{LVV-146} - DMS-REQ-0315-V-01: DMS Communication with OCS
\item \href{https://jira.lsstcorp.org/browse/LVV-115}{LVV-115} - DMS-REQ-0284-V-01: Level-1 Production Completeness
\end{itemize}

\subsubsection{Test Items}
This test will check:

\begin{itemize}
\tightlist
\item
  The successful integration of the Pathfinder components with the CCOB;
\item
  That the file can then be ingested into the DBB and be retrieved for
  further analysis;
\end{itemize}



\subsubsection{Intercase Dependencies}
None.


\subsubsection{Environment Needs}

\paragraph{Software}
\begin{itemize}
\tightlist
\item
  CCOB device and the software to produce a file to be transferred and
  kept
\item
  DBB software to produce a retrieval file for further processing
\end{itemize}


\paragraph{Hardware}
\begin{itemize}
\tightlist
\item
  CCOB
\item
  Test machine for LSST Monitoring Service
\end{itemize}


\subsubsection{Input Specification}
None.


\subsubsection{Output Specification}
\begin{itemize}
\tightlist
\item
  CCOB files that follow specifications;
\item
  DBB files that follow specifications;
\item
  CCOB device directs a human to where a file is wanted to be stored in
  the DBB;
\item
  Transfer the file to the DBB ingest area;
\end{itemize}


\subsubsection{Test Procedure}
    \begin{longtable}[]{p{1.3cm}p{2cm}p{13cm}}
    %\toprule
    Step & \multicolumn{2}{@{}l}{Description, Input Data and Expected Result} \\ \toprule
    \endhead

            \multirow{3}{*}{ 1 } & Description &
            \begin{minipage}[t]{13cm}{\footnotesize
            CCOB device directs a human to where a file is wanted to be stored in
the DBB

            \vspace{\dp0}
            } \end{minipage} \\ \cline{2-3}
            & Test Data &
            \begin{minipage}[t]{13cm}{\footnotesize
                No data.
                \vspace{\dp0}
            } \end{minipage} \\ \cline{2-3}
            & Expected Result &
                \begin{minipage}[t]{13cm}{\footnotesize
                a file has been created by the device. ~a UNIQUE file name is needed for
this result to work.~~

                \vspace{\dp0}
                } \end{minipage}
        \\ \midrule

            \multirow{3}{*}{ 2 } & Description &
            \begin{minipage}[t]{13cm}{\footnotesize
            Transfer the file to the DBB ingest area

            \vspace{\dp0}
            } \end{minipage} \\ \cline{2-3}
            & Test Data &
            \begin{minipage}[t]{13cm}{\footnotesize
                No data.
                \vspace{\dp0}
            } \end{minipage} \\ \cline{2-3}
            & Expected Result &
                \begin{minipage}[t]{13cm}{\footnotesize
                A command is executed by a human with a file name and path to the file
wanted to be stored in the DBB.~ The file is transferred to NCSA's DBB
ingest area.~ ~~

                \vspace{\dp0}
                } \end{minipage}
        \\ \midrule
    \end{longtable}

\subsection{\href{https://jira.lsstcorp.org/secure/Tests.jspa\#/testCase/LVV-T285}{LVV-T285}
    - RAS-00-10: Raw images in Observatory Operations Data Service}\label{lvv-t285}

\begin{longtable}[]{llllll}
\toprule
Version & Status & Priority & Verification Type & Critical Event & Owner
\\\midrule
1 & Draft & Normal &
Test & False & Michelle Butler
\\\bottomrule
\end{longtable}

\subsubsection{Requirements}
    None.

\subsubsection{Test Items}
This test will check:

\begin{itemize}
\tightlist
\item
  The handoff of a raw image from the Level 1 Archiver to the OODS cache
  manager is successful;
\item
  A recently taken raw image is accessible to the Observatory Operations
  staff at the base and summit;
\end{itemize}

~This Test Case shall be repeated for each of the different cameras
(ATScam, LSSTCam) and sensors (Science, Wavefront, and Guider)
combination.



\subsubsection{Intercase Dependencies}
LVV-T283


\subsubsection{Environment Needs}

\paragraph{Software}
The following software must be installed:\\
~\\

\begin{itemize}
\tightlist
\item
  Level 1 Test Stand (include software from LVV-T283 - RAS-00-00)
\item
  OODS cache manager
\item
  LSST Monitoring Service and plugins specific to monitoring raw images
  and OODS~
\item
  LSST stack for checking raw images
\end{itemize}


\paragraph{Hardware}
To complete all tests in a manner which reflects the real system, the
following hardware is needed. Note: If not testing inter-machine access,
the hardware can be minimized to a single machine outside of the Level 1
Test Stand.

\begin{itemize}
\tightlist
\item
  Level1TestStand(include hardware from LVV-T283 - RAS-00-00)+read/write
  access to OODS cache disk
\item
  Test Machine for OODS cache manager with read/write access to OODS
  cache disk
\item
  Test machine for Observatory Operations staff at ''base'' that can
  access OODS cache disk
\item
  Test machine for Observatory Operations staff at ''summit'' that can
  access OODS cache disk
\item
  Test machine for LSST Monitoring Service
\end{itemize}

Size of cache disk is determined by number of files to be included in
the test.


\subsubsection{Input Specification}

\subsubsection{Output Specification}
Raw image(s) that follow format defined in LSE-61;\\
Database (may be SQLite file) that enables the raw image(s) to be
accessed via a ``Data Butler''.


\subsubsection{Test Procedure}
    \begin{longtable}[]{p{1.3cm}p{2cm}p{13cm}}
    %\toprule
    Step & \multicolumn{2}{@{}l}{Description, Input Data and Expected Result} \\ \toprule
    \endhead

            \multirow{3}{*}{ 1 } & Description &
            \begin{minipage}[t]{13cm}{\footnotesize
            Initialize all services configuring the Level 1 Archiver Service so that
the raw images are to be saved to the OODS

            \vspace{\dp0}
            } \end{minipage} \\ \cline{2-3}
            & Test Data &
            \begin{minipage}[t]{13cm}{\footnotesize
                No data.
                \vspace{\dp0}
            } \end{minipage} \\ \cline{2-3}
            & Expected Result &
                \begin{minipage}[t]{13cm}{\footnotesize
                all camera and services for images are running and reporting green
through the monitoring programs for the services. ~\\
~\\

                \vspace{\dp0}
                } \end{minipage}
        \\ \midrule

            \multirow{3}{*}{ 2 } & Description &
            \begin{minipage}[t]{13cm}{\footnotesize
            Acquire a raw image

            \vspace{\dp0}
            } \end{minipage} \\ \cline{2-3}
            & Test Data &
            \begin{minipage}[t]{13cm}{\footnotesize
                No data.
                \vspace{\dp0}
            } \end{minipage} \\ \cline{2-3}
            & Expected Result &
                \begin{minipage}[t]{13cm}{\footnotesize
                Image present in the input folder.

                \vspace{\dp0}
                } \end{minipage}
        \\ \midrule

            \multirow{3}{*}{ 3 } & Description &
            \begin{minipage}[t]{13cm}{\footnotesize
            \emph{The handoff of the raw image from the Level 1 Archiver Service to
the test OODS automatically occurs\\
}

            \vspace{\dp0}
            } \end{minipage} \\ \cline{2-3}
            & Test Data &
            \begin{minipage}[t]{13cm}{\footnotesize
                No data.
                \vspace{\dp0}
            } \end{minipage} \\ \cline{2-3}
            & Expected Result &
                \begin{minipage}[t]{13cm}{\footnotesize
                the raw image with a proper header is written to a file area managed by
the OODS\\
~\\

                \vspace{\dp0}
                } \end{minipage}
        \\ \midrule

            \multirow{3}{*}{ 4 } & Description &
            \begin{minipage}[t]{13cm}{\footnotesize
            For each of the expected raw images, verify that the checksum matches
the original Level 1 checksum

            \vspace{\dp0}
            } \end{minipage} \\ \cline{2-3}
            & Test Data &
            \begin{minipage}[t]{13cm}{\footnotesize
                No data.
                \vspace{\dp0}
            } \end{minipage} \\ \cline{2-3}
            & Expected Result &
                \begin{minipage}[t]{13cm}{\footnotesize
                checksum of the file is checked against the file for verification that
the OODS has the correct file and it matches the original md5sum of the
FITS file.\\
~\\

                \vspace{\dp0}
                } \end{minipage}
        \\ \midrule

            \multirow{3}{*}{ 5 } & Description &
            \begin{minipage}[t]{13cm}{\footnotesize
            Check that LSST Monitoring Service showed the appropriate information
successfully

            \vspace{\dp0}
            } \end{minipage} \\ \cline{2-3}
            & Test Data &
            \begin{minipage}[t]{13cm}{\footnotesize
                No data.
                \vspace{\dp0}
            } \end{minipage} \\ \cline{2-3}
            & Expected Result &
                \begin{minipage}[t]{13cm}{\footnotesize
                Make sure all camera and OODS systems were available thorughout this
test.~ ~

                \vspace{\dp0}
                } \end{minipage}
        \\ \midrule
    \end{longtable}

\subsection{\href{https://jira.lsstcorp.org/secure/Tests.jspa\#/testCase/LVV-T286}{LVV-T286}
    - RAS-00-20: Raw image are part of the permanent record of survey via DBB}\label{lvv-t286}

\begin{longtable}[]{llllll}
\toprule
Version & Status & Priority & Verification Type & Critical Event & Owner
\\\midrule
1 & Draft & Normal &
Test & False & Michelle Butler
\\\bottomrule
\end{longtable}

\subsubsection{Requirements}
\begin{itemize}
\item \href{https://jira.lsstcorp.org/browse/LVV-28}{LVV-28} - DMS-REQ-0068-V-01: Raw Science Image Metadata
\item \href{https://jira.lsstcorp.org/browse/LVV-177}{LVV-177} - DMS-REQ-0346-V-01: Data Availability
\item \href{https://jira.lsstcorp.org/browse/LVV-115}{LVV-115} - DMS-REQ-0284-V-01: Level-1 Production Completeness
\end{itemize}

\subsubsection{Test Items}
This test will check:\\
~\\

\begin{itemize}
\tightlist
\item
  That the handoff of a raw image from the Level 1 Archiver Service to
  the DBB buffer manager is successful;
\item
  That the raw image is ingested into the Data Backbone successfully;
\item
  That the monitoring of the above items is successful;
\end{itemize}

This Test Case shall be repeated for each of the different cameras
(ATScam, LSSTCam) and sensors (Science, Wavefront, and Guider)
combination.\\
~\\
Note: For a complete check of the various aspects of what it means for a
raw image to be in the Data Backbone, see the tests for the Data
Backbone.



\subsubsection{Intercase Dependencies}
LVV-T283


\subsubsection{Environment Needs}

\paragraph{Software}
\begin{itemize}
\tightlist
\item
  Level 1 Test Stand
\item
  DBB buffer manager
\item
  DBB raw image ingestion
\item
  DBB database
\item
  LSST Monitoring Service and plugins specific to monitoring raw images,
  DBB buffer manager, and DBB
\end{itemize}


\paragraph{Hardware}
\begin{itemize}
\tightlist
\item
  Level 1 Test Stand (include hardware from LVV-T-283 - RAS-00-00) +
  read/write access to DBB buffer disk;
\item
  Test Machine for DBB buffer manager with read/write access to DBB
  buffer disk;
\item
  Test machine for each DBB endpoint with read/write access to DBB disk;
\item
  Test machine for LSST Monitoring Service
\end{itemize}

Size of buffer disk and DBB disk is determined by number of files to be
included in the test.\\
~\\
Note: If not testing inter-machine operability, then the hardware can be
minimized to a single machine outside of the Level 1 test stand.


\subsubsection{Input Specification}
​​​​​None


\subsubsection{Output Specification}
\begin{itemize}
\tightlist
\item
  Raw image(s) are saved to storage and replicated to correct locations
  with checksums that match original Level 1 checksum;
\item
  Database containing information of the following types: physical,
  location, science metadata, provenance as specified in LSE-61;
\item
  Both image(s) and database entries replicated correctly;
\end{itemize}


\subsubsection{Test Procedure}
    \begin{longtable}[]{p{1.3cm}p{2cm}p{13cm}}
    %\toprule
    Step & \multicolumn{2}{@{}l}{Description, Input Data and Expected Result} \\ \toprule
    \endhead

            \multirow{3}{*}{ 1 } & Description &
            \begin{minipage}[t]{13cm}{\footnotesize
            Initialize all services configuring the Level 1 Archiver Service so that
the raw images are to be archived to the DBB

            \vspace{\dp0}
            } \end{minipage} \\ \cline{2-3}
            & Test Data &
            \begin{minipage}[t]{13cm}{\footnotesize
                No data.
                \vspace{\dp0}
            } \end{minipage} \\ \cline{2-3}
            & Expected Result &
                \begin{minipage}[t]{13cm}{\footnotesize
                all services for the camera images and the DBB services are all running
and ready for data.~~

                \vspace{\dp0}
                } \end{minipage}
        \\ \midrule

            \multirow{3}{*}{ 2 } & Description &
            \begin{minipage}[t]{13cm}{\footnotesize
            Acquire a raw image (see LVV-T283 - RAS-00-00){\\
}

            \vspace{\dp0}
            } \end{minipage} \\ \cline{2-3}
            & Test Data &
            \begin{minipage}[t]{13cm}{\footnotesize
                No data.
                \vspace{\dp0}
            } \end{minipage} \\ \cline{2-3}
            & Expected Result &
                \begin{minipage}[t]{13cm}{\footnotesize
                have a raw Fits file with proper header.~~

                \vspace{\dp0}
                } \end{minipage}
        \\ \midrule

            \multirow{3}{*}{ 3 } & Description &
            \begin{minipage}[t]{13cm}{\footnotesize
            After the automatic handoff of the raw image between the Level 1
Archiver Service and the DBB buffer manager, the raw image will
automatically be ingested into the Data Backbone

            \vspace{\dp0}
            } \end{minipage} \\ \cline{2-3}
            & Test Data &
            \begin{minipage}[t]{13cm}{\footnotesize
                No data.
                \vspace{\dp0}
            } \end{minipage} \\ \cline{2-3}
            & Expected Result &
                \begin{minipage}[t]{13cm}{\footnotesize
                the DBB file systems will have the file, and metadata and providence
will be recorded in the consolidated DB. ~ The file will also be
replicated to mulitple locations for DR.~~

                \vspace{\dp0}
                } \end{minipage}
        \\ \midrule

            \multirow{3}{*}{ 4 } & Description &
            \begin{minipage}[t]{13cm}{\footnotesize
            Check that the raw image is accessible at each ~DBB endpoint and matches
original Level 1 checksum

            \vspace{\dp0}
            } \end{minipage} \\ \cline{2-3}
            & Test Data &
            \begin{minipage}[t]{13cm}{\footnotesize
                No data.
                \vspace{\dp0}
            } \end{minipage} \\ \cline{2-3}
            & Expected Result &
                \begin{minipage}[t]{13cm}{\footnotesize
                data resides at NCSA DBB end point, and Chile end point and match with
the same checksum.~~

                \vspace{\dp0}
                } \end{minipage}
        \\ \midrule

            \multirow{3}{*}{ 5 } & Description &
            \begin{minipage}[t]{13cm}{\footnotesize
            Check that LSST Monitoring Service showed the appropriate information
successfully

            \vspace{\dp0}
            } \end{minipage} \\ \cline{2-3}
            & Test Data &
            \begin{minipage}[t]{13cm}{\footnotesize
                No data.
                \vspace{\dp0}
            } \end{minipage} \\ \cline{2-3}
            & Expected Result &
                \begin{minipage}[t]{13cm}{\footnotesize
                all related systems remained up during this test.~~

                \vspace{\dp0}
                } \end{minipage}
        \\ \midrule

            \multirow{3}{*}{ 6 } & Description &
            \begin{minipage}[t]{13cm}{\footnotesize
            More complete tests of the DBB can be done by running the DBB service
tests on the raw image(s). These would check correctness and
completeness of the data stored in the database as well as checking that
the file has been replicated to all required places

            \vspace{\dp0}
            } \end{minipage} \\ \cline{2-3}
            & Test Data &
            \begin{minipage}[t]{13cm}{\footnotesize
                No data.
                \vspace{\dp0}
            } \end{minipage} \\ \cline{2-3}
            & Expected Result &
                \begin{minipage}[t]{13cm}{\footnotesize
                These would be more tests of when things go wrong to make sure that the
DBB is able to continue to work, and not be in the way of taking images
from the camera\\
~\\

                \vspace{\dp0}
                } \end{minipage}
        \\ \midrule
    \end{longtable}

\subsection{\href{https://jira.lsstcorp.org/secure/Tests.jspa\#/testCase/LVV-T287}{LVV-T287}
    - RAS-00-30: Raw Image Archiving Availability, Throughput, Reliability, and Heterogeneity}\label{lvv-t287}

\begin{longtable}[]{llllll}
\toprule
Version & Status & Priority & Verification Type & Critical Event & Owner
\\\midrule
1 & Draft & Normal &
Test & False & Michelle Butler
\\\bottomrule
\end{longtable}

\subsubsection{Requirements}
\begin{itemize}
\item \href{https://jira.lsstcorp.org/browse/LVV-5}{LVV-5} - DMS-REQ-0008-V-01: Pipeline Availability
\item \href{https://jira.lsstcorp.org/browse/LVV-65}{LVV-65} - DMS-REQ-0162-V-01: Pipeline Throughput
\item \href{https://jira.lsstcorp.org/browse/LVV-68}{LVV-68} - DMS-REQ-0165-V-01: Infrastructure Sizing for "catching up"
\item \href{https://jira.lsstcorp.org/browse/LVV-70}{LVV-70} - DMS-REQ-0167-V-01: Incorporate Autonomics
\item \href{https://jira.lsstcorp.org/browse/LVV-145}{LVV-145} - DMS-REQ-0314-V-01: Compute Platform Heterogeneity
\item \href{https://jira.lsstcorp.org/browse/LVV-149}{LVV-149} - DMS-REQ-0318-V-01: Data Management Unscheduled Downtime
\item \href{https://jira.lsstcorp.org/browse/LVV-140}{LVV-140} - DMS-REQ-0309-V-01: Raw Data Archiving Reliability
\end{itemize}

\subsubsection{Test Items}
This test will check:\\
~\\

\begin{itemize}
\tightlist
\item
  Raw Image Archiving meets availability requirements;
\item
  Raw Image Archiving meets throughput requirements;
\item
  Raw Image Archiving meets reliability requirements;
\item
  Raw Image Archiving meets heterogeneity requirements;
\end{itemize}

This test case need to be completed when more information is available.



\subsubsection{Intercase Dependencies}

\subsubsection{Environment Needs}

\paragraph{Software}

\paragraph{Hardware}

\subsubsection{Input Specification}

\subsubsection{Output Specification}

\subsubsection{Test Procedure}
    \begin{longtable}[]{p{1.3cm}p{2cm}p{13cm}}
    %\toprule
    Step & \multicolumn{2}{@{}l}{Description, Input Data and Expected Result} \\ \toprule
    \endhead

            \multirow{3}{*}{ 1 } & Description &
            \begin{minipage}[t]{13cm}{\footnotesize
            these will be filled out as the service becomes more known as to what
the availablility, throughput, reliability and heterogeneity are. ~\\
~\\

            \vspace{\dp0}
            } \end{minipage} \\ \cline{2-3}
            & Test Data &
            \begin{minipage}[t]{13cm}{\footnotesize
                No data.
                \vspace{\dp0}
            } \end{minipage} \\ \cline{2-3}
            & Expected Result &
                \begin{minipage}[t]{13cm}{\footnotesize
                The archive system will stay up through thick and thin and perform like
it's suppose to.\\
~\\

                \vspace{\dp0}
                } \end{minipage}
        \\ \midrule
    \end{longtable}

