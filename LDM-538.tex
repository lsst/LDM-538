\documentclass[DM,lsstdraft,STS,toc]{lsstdoc}
% lsstdoc documentation: https://lsst-texmf.lsst.io/lsstdoc.html
% Package imports go here.


\usepackage{enumitem}
\begin{document}
\def\product{Data Management - LSST Raw Image Archiving Service}
\setDocCompact{true}
\author{Michelle Butler, Jim Parsons, Michelle Gower }
\setDocDate{\vcsdate}
\setDocRef{LDM-538}


\date{\today}


% Change history defined here.
\setDocChangeRecord{%
\addtohist{1}{2018-04-18}{initial release - draft version}{Michelle Butler }
}


\title[Test Spec for \product]{\product~Test Specification}




\setDocAbstract {
This document describes the detailed test specification for the \product{}. This is a specific DM test, and will grow as more tests
are needed for the entire environment. This includes two individual
tests for the overall raw image creation and ingest into the permanent
record of the survey.
}


\setDocUpstreamLocation{\url{https://github.com/lsst/ldm-538}}
% \setDocUpstreamVersion{\vcsrevision}


\maketitle


\section{Introduction}
\label{sec:intro}


This document specifies the test procedure for the \product{}.


The \product{} is the component of the LSST system which is responsible for:
\begin{itemize}
\item{The creation of a well-formed raw image;}
\item{Providing fast access to the raw image by the Observatory Operations staff;}
\item{Saving the raw image into the permanent record of the survey;}
\end{itemize}


A full description of this service is provided in \S5.1.1 of \citeds{LDM-148} and \S2.2 (which describes LSSTCam Archiving Service), \S2.3 (Spectrograph Archiving Service) and \S2.6 (Observatory Operations Data Service) of \citeds{LDM-230}.



\subsection{Objectives}
\label{sec:objectives}




This document builds on the description of LSST Data Management's
approach to testing as described in \citeds{LDM-503} to describe
the detailed tests that will be performed on the \product{} as part
of the verification of the DM system.


It identifies test designs, test cases and procedures for the tests,
and the pass/fail criteria for each test.

\subsection{Definitions, Acronyms, and Abbreviations \label{sect:acronyms}}
\addtocounter{table}{-1}
\begin{longtable}{|l|p{0.8\textwidth}|}\hline
\textbf{Acronym} & \textbf{Description} \\\hline
CCS&Camera Control System \\\hline
DAQ&Data Acquisition\\\hline
OODS&Observatory ?? Data System\\\hline
DBB&Data Backbone \\\hline
DMHS&Data Management Header Service \\\hline
DR&Disaster Recovery \\\hline
L1&Level One Image Ingest and Data Distribution Environment \\\hline
NCSA&National Center for Supercomputing \\\hline
OODS&Observatory Operations Data Service \\\hline
OCS&Observatory Control System \\\hline
\end{longtable}


\subsection{Scope}
\label{sec:scope}


This document describes the test procedures for the Raw Image Archiving Service
which includes parts from each of the following components of the LSST system
(as described in \citeds{LDM-148}):


\begin{itemize}

\item{OCS}
\item{Camera DAQ)}
\item{DM Header Service}
\item{L1 Archiver}
\item{EFD Large File Annex}
\item{DM OCS Bridge}
\item{DMCS}
\item{Archiver}
\item{Catch-up Archiver}
\item{Alert Processor}
\item{ATS achiver}
\item {Data Forwarders}
\item{Observatory Operations Data Service}
\item{Data Backbone Services}


\end{itemize}


\subsection{Applicable Documents}
\label{sec:docs}


\addtocounter{table}{-1}


\begin{tabular}[htb]{l l}


\citeds{LSE-209} & Software Component to Observatory Control System (OCS) Interface \\
% \citeds{LSE-73} & Observatory Control System (OCS) to Telescope Software Communication Interface \\
\citeds{LSE-68} & Camera Data Acquisition Interface \\
\citeds{LSE-70} & System Communication Protocol Interface \\
\citeds{LSE-72} & Data Management - OCS Software Communication Interface \\
\citeds{LDM-148} & LSST DM System Architecture \\
\citeds{LDM-294} & LSST DM Organization \& Management \\
\citeds{LDM-503} & LSST DM Test Plan \\
\citeds{LSE-61} & LSST DM Subsystem Requirements \\
\citeds{LSE-163} & LSST Data Products Definition Document \\
\citeds{LSE-29} & LSST Data Products Definition Document \\


\end{tabular}


\subsection{References\label{sect:references}}
\renewcommand{\refname}{}
\bibliography{lsst,refs,books,refs_ads}




%----------------------------------------------------
% TASK IDENTIFICATION - APPROACH
%----------------------------------------------------
\section{Approach}
\label{sec:approach}


The major activities to be performed are to:
\begin{itemize}
\item{Show successful integration between Teloscope subsystems and L1 Archiver service;}
\item{Show successful integration between L1 Archiver and OODS;}
\item{Show successful integration between L1 Archiver and DBB;}
\end{itemize}




\subsection{Tasks and criteria}
\label{sec:tasks}


The following are the major items under test:


\begin{itemize}
\item{Create a well-formed raw image with data acquired from the Camera DAQ and the DM Header Service; }
\item{Save a raw image, with matching L1 checksum, in the OODS cache and access it using normal LSST mechanisms (e.g., ``Data Butler''); }
\item{Ingest a raw image, with matching L1 checksum, into the DBB; }
\end{itemize}




\subsection{Features to be tested}
\label{sec:feat2test}


Do the following satisfy the requirements described in \citeds{LSE-61}?
\begin{itemize}
\item{Proper fetch and reassembly of image data from camera DAQ;}
\item{Proper merge of header service data with image data;}
\item{Correct insertion of exposure specific data needed in the data file that is not suppolied by header service;}
\item{Confirmation that the data files arrive at their destination intact;}
\item{Raw data access;}
\item{Archiving of raw images;}
\end{itemize}


\subsection{Features not to be tested}
\label{sec:featnot2test}


This document describes the end-to-end testing of various components
that comprise the service. It does not include tests internal to
a single component. Also this test specification does not
extend beyond the positive pass/fail options. It does not take
into account a failure of a component and how the subsystem should
recover from those failures.


\subsection{Pass/fail criteria}
\label{sec:passfail}


The results of all tests will be assessed using the criteria described in
\citeds{LDM-503} \S4.




\subsection{Suspension criteria and resumption requirements}
\label{suspension}


Refer to individual test cases where applicable.


%----------------------------------------------------
% TASK IDENTIFICATION - Test Specification Design
%----------------------------------------------------
\section{Test Specification Design}
\label{sec:Test Specification Design}
\subsection{RAS-00: Raw Image Archive Service}


\subsubsection{Objective}
This test specification demonstrates the successful writing of a
well-formed raw image to the permanent record of the survey and
for rapid access for Observatory Operations staff.


\subsubsection{Test case identification}


\begin{longtable} {|p{0.4\textwidth}|p{0.6\textwidth}|}\hline
\textbf{Test Case} & \textbf{Description} \\\hline
\hyperref[ras-00-00]{RAS-00-00} & Tests that well-formed raw image can be written \\\hline
\hyperref[ras-00-10]{RAS-00-10} & Tests that the raw image can be accessed via the OODS \\\hline
\hyperref[ras-00-20]{RAS-00-20} & Tests that the raw image has been archived in the DBB \\\hline
\end{longtable}


\section{Test Case Specification}


%----------------------------------------------------
\subsection{RAS-00-00: Writing well-formed raw image}
\label{ras-00-00}


\subsubsection{Requirements}
\begin{itemize}
\item{;}

\item {Data must be fetched and reassembled correctly, regardless of CCD/Sensor manufacturer type (two different types will be used);}
\item{The base unit of data to be saved is by the amplifier segments of each CCD/sensor;}
\item{After Fits files are built but before they are sent to an archive location, their checksum should be computed.;}
\item{Files are verified after being moved to an archive using the checksum value;}
\item{Upon shutdown, the L1 Image Ingest and Distribution system should issue a report stating which data (by Image-ID) was processed and whether it was successfully handed off to archive services.;}
\end{itemize}

\subsubsection{Test items}
This test will check:


\begin{itemize}
\item{The successful integration of the Pathfinder components with the DM Header Service and the L1 Archiver;}
\item{That the raw images are well-formed and meet specifications in change-controlled documents \citeds{LSE-61} and TBD.}
\end{itemize}


\subsubsection{Intercase dependencies}
\begin{itemize}
\item{None}
\end{itemize}


\subsubsection{Environmental needs}
\paragraph{Hardware}
\begin{itemize}
\item{L1 test stand}
\item{Test machine for monitoring services}
\end{itemize}


\paragraph{Software}
\begin{itemize}
\item{L1 software and services needed to create raw image}
\item{A SAL message Emulator to represent needed messages from other LSST subsystems}
\item{Monitoring Service and plugins specific to monitoring L1 Test Stand and services}
\end{itemize}


\subsubsection{Input specification}
None
\subsubsection{Output specification}
\begin{itemize}
\item{Raw image(s) that follow specifications defined in change-controlled document TBD.}
\end{itemize}


\subsubsection{Procedure}
\begin{itemize}
\item{Configure system to pull appropriate data from the DAQ emulator}
\item{Aquire system to create raw image from DAQ readout and DMHS}
\item{Check raw image against specifications and expected checksum}
\end{itemize}




%----------------------------------------------------
\subsection{RAS-00-10: Raw images in Observatory Operations Data Service}
\label{ras-00-10}
\subsubsection{Requirements}
Requirements for the OODS
\begin{itemize}
\item{the handoff of the data;}
\item{that the data is be accessible;}
% ** note Perhaps there should be a bullet stating that someone (My code?) will send a SAL event stating that an image is available to the OODS and includes information such as Image ID, Visit Number, Filter used, etc.

\end{itemize}




\subsubsection{Test items}
This test will check:
\begin{itemize}
\item{The handoff of a raw image from the L1 Archiver to the OODS cache
manager is successful}
\item{A recently taken raw image is accessible to the Observatory
Operations staff at the base and summit}
\end{itemize}


\subsubsection{Intercase dependencies}
\begin{itemize}
\item{\hyperref[ras-00-00]{RAS-00-00}}
\end{itemize}


\subsubsection{Environmental needs}
\paragraph{Hardware}


To complete all tests in a manner which reflects the real system,
the following hardware is needed.  Note: If not testing inter-machine access, the hardware can be minimized to a single machine outside of the L1 Test Stand.


\begin{itemize}
\item{L1 Test Stand (include hardware from \hyperref[ras-00-00]{RAS-00-00}) + read/write access to OODS cache disk}
\item{Test Machine for OODS cache manager with read/write access to OODS cache disk}
\item{Test machine for Observatory Operations staff at "base" that can access OODS cache disk}
\item{Test machine for Observatory Operations staff at "summit" that can access OODS cache disk}
\item{Test machine for monitoring services}
\end{itemize}


Size of cache disk is determined by number of files to be included in the test.






\paragraph{Software}


The following software must be installed:
\begin{itemize}
\item{L1 Test Stand (include software from \hyperref[ras-00-00]{RAS-00-00})}
\item{OODS cache manager}
\item{Monitoring Service and plugins specific to monitoring raw images and OODS}
\item{LSST stack for checking raw images}
\end{itemize}




\subsubsection{Input specification}


None.


\subsubsection{Output specification}
\begin{itemize}
\item{Raw image(s) that follow format defined in change-controlled document TBD.}
\item{Database (may be sqlite file) that enables the raw image(s) to be accessed via a ``Data Butler''.}
\end{itemize}


\subsubsection{Procedure}


\begin{itemize}
\item{Initialize all services configuring the L1 Archiver so that the raw images are to be archived}
\item{Aquire an image. After the raw image finishes its route through the components, it
will automatically be in the Test Data Backbone.}
\item{For each of the expected raw images, verify that the checksum matches the original L1 checksum.}
\item{The DM Stack shall be initialized using the \texttt{loadLSST} script}
\item{A ``Data Butler'' will be initialized to access the raw image repository in the OODS cache.}
\item{For each of the expected raw images, the file will be retrieved from the ``Data Butler'' and verified to meet the raw image requirements (correctness and completeness of format, metadata and image data) as specified in change-controlled document (TBD)}
\item{Check that monitoring showed the appropriate information successfully.}
\end{itemize}




%----------------------------------------------------
\subsection{RAS-00-20: Raw images are part of permanent record of survey via DBB}
\label{ras-00-20}


\subsubsection{Requirements}


\begin{itemize}
\item{LSR-REQ-0047}
\item{LSR-REQ-0048}
\end{itemize}


\subsubsection{Test items}


This test will check:
\begin{itemize}
\item{That the handoff of a raw image from the L1 Archiver to the DBB buffer
manager is successful}
\item{That the raw image is ingested into the Data Backbone successfully}
\item{That the monitoring of the above items is successful}
\end{itemize}


Note: For a complete check of the various aspects of what it means for
a raw image to be in the Data Backbone, see the tests for the Data Backbone (doc TBD)


\subsubsection{Intercase dependencies}
\begin{itemize}
\item{\hyperref[ras-00-00]{RAS-00-00}}
\end{itemize}


\subsubsection{Environmental needs}
\paragraph{Hardware}
\begin{itemize}
\item{L1 Test Stand (include hardware from \hyperref[ras-00-00]{RAS-00-00}) + read/write access to DBB buffer disk}
\item{Test Machine for DBB buffer manager with read/write access to DBB buffer disk}
\item{Test machine for each DBB endpoint with read/write access to DBB disk}
\item{Test machine for monitoring service}
\end{itemize}


Size of buffer disk and DBB disk is determined by number of files to be included in the test.


Note: If not testing inter-machine operability, then the hardware can be minimized
to a single machine outside of the L1 test stand.


\paragraph{Software}
\begin{itemize}
\item{L1 Test Stand}
\item{DBB buffer manager}
\item{DBB raw image ingestion}
\item{DBB database}
\item{Monitoring Service and plugins specific to monitoring raw images, DBB buffer manager, and DBB}
\end{itemize}




\subsubsection{Input specification}
None


\subsubsection{Output specification}
\begin{itemize}
\item{Raw image(s) are saved to storage and replicated to correct locations with checksums that match
original L1 checksum.}
\item{Database containing information of the following types: physical, location, science metadata,
provenance as specified in change control document (TBD).}
\item{Both image(s) and database entries replicated correctly.}
\end{itemize}


\subsubsection{Procedure}
\begin{itemize}
\item{Initialize all services configuring the L1 Archiver so that the raw images are to be archived.}
\item{Aquire an image to be taken. After the raw image finishes its route through the components, it
will automatically be in the Test Data Backbone.}
\item{Check that the raw image is accessible at each test DBB endpoint and matches original L1 checksum.}
\item{Check that monitoring showed the appropriate information successfully.}
\item{More complete tests of the DBB can be done by running the DBB
service tests on the raw image(s). These would check correctness
and completeness of the data stored in the database as well as
checking that the file has been replicated to all required places.}
\end{itemize}


\end{document}
